@article{AbadieAlberto2015CPat,
abstract = {In recent years, a widespread consensus has emerged about the necessity of establishing bridges between quantitative and qualitative approaches to empirical research in political science. In this article, we discuss the use of the synthetic control method as a way to bridge the quantitative/qualitative divide in comparative politics. The synthetic control method provides a systematic way to choose comparison units in comparative case studies. This systematization opens the door to precise quantitative inference in small-sample comparative studies, without precluding the application of qualitative approaches. Borrowing the expression from Sidney Tarrow, the synthetic control method allows researchers to put "qualitative flesh on quantitative bones." We illustrate the main ideas behind the synthetic control method by estimating the economic impact of the 1990 German reunification on West Germany.},
language = {eng},
number = {2},
pages = {495-510},
year = {2015},
address = {HOBOKEN},
copyright = {2015 Midwest Political Science Association},
author = {Abadie, Alberto and Diamond, Alexis and Hainmueller, Jens},
issn = {0092-5853},
journal = {American journal of political science},
keywords = {AJPS WORKSHOP ; Comparative analysis ; comparative case studies ; Comparative Politics ; Contrafactuals ; difference-in-differences ; Economic impact ; Economic impact analysis ; Empirical Methods ; Empirical research ; Estimation methods ; Federal Republic of Germany ; German Reunification ; Germany (West) ; Government & Law ; Gross domestic product ; Industry ; Inference ; Linear regression ; matching ; Outcome variables ; Placebos ; Political Science ; Qualitative analysis ; Qualitative Methods ; Quantitative analysis ; Quantitative Methods ; Reunification ; Social Sciences ; synthetic control method ; Trade and industrial education ; Weighted averages},
publisher = {Blackwell Publishing Ltd},
title = {Comparative Politics and the Synthetic Control Method},
volume = {59},
}
@article{AbadieAlberto2010SCMf,
abstract = {Building on an idea in Abadie and Gardeazabal (2003), this article investigates the application of synthetic control methods to comparative case studies. We discuss the advantages of these methods and apply them to study the effects of Proposition 99, a large-scale tobacco control program that California implemented in 1988. We demonstrate that, following Proposition 99, tobacco consumption fell markedly in California relative to a comparable synthetic control region. We estimate that by the year 2000 annual per-capita cigarette sales in California were about 26 packs lower than what they would have been in the absence of Proposition 99. Using new inferential methods proposed in this article, we demonstrate the significance of our estimates. Given that many policy interventions and events of interest in social sciences take place at an aggregate level (countries, regions, cities, etc.) and affect a small number of aggregate units, the potential applicability of synthetic control methods to comparative case studies is very large, especially in situations where traditional regression methods are not appropriate.},
language = {eng},
number = {490},
pages = {493-505},
year = {2010},
address = {ALEXANDRIA},
copyright = {2010 American Statistical Association},
author = {Abadie, Alberto and Diamond, Alexis and Hainmueller, Jens},
issn = {0162-1459},
journal = {Journal of the American Statistical Association},
keywords = {Anti smoking movements ; Applications ; Applications and Case Studies ; Biology, psychology, social sciences ; Case studies ; Cigarette smoking ; Cigarettes ; Estimation methods ; Estimators ; Exact sciences and technology ; General topics ; Geometry ; Inference ; Laws, regulations and rules ; Legislation ; Linear inference, regression ; Mathematics ; Physical Sciences ; Placebos ; Probability and statistics ; Public policy ; Science & Technology ; Sciences and techniques of general use ; Smoking cessation ; Statistics ; Statistics & Probability ; Tobacco industry ; Tobacco products ; Tobacco taxes},
publisher = {American Statistical Association},
title = {Synthetic Control Methods for Comparative Case Studies: Estimating the Effect of California's Tobacco Control Program},
volume = {105},
}
@article{AbadieA2003TECo,
abstract = {This article investigates the economic effects of conflict, using the terrorist conflict in the Basque Country as a case study. We find that, after the outbreak of terrorism in the late 1960's, per capita GDP in the Basque Country declined about 10 percentage points relative to a synthetic control region without terrorism. In addition, we use the 1998-1999 truce as a natural experiment. We find that stocks of firms with a significant part of their business in the Basque Country showed a positive relative performance when truce became credible, and a negative relative performance at the end of the cease-fire.},
language = {eng},
number = {1},
pages = {113-132},
year = {2003},
address = {NASHVILLE},
copyright = {Copyright 2003 American Economic Association},
author = {Abadie, A and Gardeazabal, J},
issn = {0002-8282},
journal = {The American economic review},
keywords = {Basque Country ; Business & Economics ; Business structures ; Case studies ; Ceasefires ; Conflict ; Countries ; Economic conditions ; Economic costs ; Economic growth ; Economic impact ; Economic regions ; Economics ; Extortion ; Financial portfolios ; GDP ; Gross Domestic Product ; Influence ; Murders & murder attempts ; Output gaps ; Per capita ; Regression analysis ; Social Sciences ; Stock shares ; Studies ; Terrorism ; Truces & cease fires},
publisher = {American Economic Association},
title = {The Economic Costs of Conflict: A Case Study of the Basque Country},
volume = {93},
}
@article{LivingstonMelvinD2017RCLa,
abstract = {Objectives. To examine the association between Colorado's legalization of recreational cannabis use and opioid-related deaths.
Methods. We used an interrupted time-series design (2000-2015) to compare changes in level and slope of monthly opioid-related deaths before and after Colorado stores began selling recreational cannabis. We also describe the percent change in opioid-related deaths by comparing the unadjusted model-smoothed number of deaths at the end of follow-up with the number of deaths just prior to legalization.
Results. Colorado's legalization of recreational cannabis sales and use resulted in a 0.7 deaths per month (b = -0.68; 95% confidence interval = -1.34, -0.03) reduction in opioid-related deaths. This reduction represents a reversal of the upward trend in opioid-related deaths in Colorado.
Conclusions. Legalization of cannabis in Colorado was associated with short-term reductions in opioid-related deaths. As additional data become available, research should replicate these analyses in other states with legal recreational cannabis.},
language = {eng},
number = {11},
pages = {1827-1829},
year = {2017},
address = {WASHINGTON},
copyright = {Copyright American Public Health Association Nov 2017},
author = {Livingston, Melvin D and Barnett, Tracey E and Delcher, Chris and Wagenaar, Alexander C},
issn = {0090-0036},
journal = {American journal of public health (1971)},
keywords = {Abridged Index Medicus ; AJPH Research ; Alcohol ; Cannabis ; Colorado - epidemiology ; Confidence intervals ; Deaths ; Disease control ; Drug abuse ; Drug legalization ; Drug Overdose - mortality ; Drugs ; Emergency medical care ; Fatalities ; Health Law ; Humans ; Injury prevention ; Life Sciences & Biomedicine ; Marijuana ; Marijuana Smoking - legislation & jurisprudence ; Medical marijuana ; Medical research ; Mortality ; Narcotics ; Opioid-Related Disorders - mortality ; Opioids ; Poisoning ; Prescription drugs ; Prevention ; Public health ; Public, Environmental & Occupational Health ; Recreational use ; Reduction ; Research ; Researchers ; Sales ; Science & Technology ; Studies ; Time series ; Trends ; Trust},
publisher = {AMER PUBLIC HEALTH ASSOC INC},
title = {Recreational Cannabis Legalization and Opioid-Related Deaths in Colorado, 2000-2015},
volume = {107},
}
@article{Salomonsen-SautelStacy2014Tifm,
abstract = {Abstract Background Legal medical marijuana has been commercially available on a widespread basis in Colorado since mid-2009; however, there is a dearth of information about the impact of marijuana commercialization on impaired driving. This study examined if the proportions of drivers in a fatal motor vehicle crash who were marijuana-positive and alcohol-impaired, respectively, have changed in Colorado before and after mid-2009 and then compared changes in Colorado with 34 non-medical marijuana states (NMMS). Methods Thirty-six 6-month intervals (1994–2011) from the Fatality Analysis Reporting System were used to examine temporal changes in the proportions of drivers in a fatal motor vehicle crash who were alcohol-impaired (≥0.08 g/dl) and marijuana-positive, respectively. The pre-commercial marijuana time period in Colorado was defined as 1994–June 2009 while July 2009–2011 represented the post-commercialization period. Results In Colorado, since mid-2009 when medical marijuana became commercially available and prevalent, the trend became positive in the proportion of drivers in a fatal motor vehicle crash who were marijuana-positive (change in trend, 2.16 (0.45), p < 0.0001); in contrast, no significant changes were seen in NMMS. For both Colorado and NMMS, no significant changes were seen in the proportion of drivers in a fatal motor vehicle crash who were alcohol-impaired. Conclusions Prevention efforts and policy changes in Colorado are needed to address this concerning trend in marijuana-positive drivers. In addition, education on the risks of marijuana-positive driving needs to be implemented.},
language = {eng},
pages = {137-144},
year = {2014},
address = {CLARE},
copyright = {Elsevier Ireland Ltd},
author = {Salomonsen-Sautel, Stacy and Min, Sung-Joon and Sakai, Joseph T and Thurstone, Christian and Hopfer, Christian},
issn = {0376-8716},
journal = {Drug and alcohol dependence},
keywords = {Accidents, Traffic - mortality ; Accidents, Traffic - trends ; Alcohol Drinking - epidemiology ; Alcohol-impaired driving ; Cannabis ; Colorado - epidemiology ; Commercialization ; Drivers ; Driving ; Drugged driving ; Humans ; Legislation, Drug - trends ; Life Sciences & Biomedicine ; Marijuana Smoking - epidemiology ; Marijuana-positive driving ; Medical Marijuana ; Motor vehicles ; Prevalence ; Preventive programmes ; Psychiatry ; Science & Technology ; Substance Abuse ; Traffic accidents ; Traffic fatalities},
publisher = {Elsevier Ireland Ltd},
title = {Trends in fatal motor vehicle crashes before and after marijuana commercialization in Colorado},
volume = {140},
}
@techreport { ipums ,
author = {Ruggles, Steven and Flood, Sarah and Foster, Sophia and Goeken, Ronald and Pacas, Jose and Schouweiler, Megan  and Sobek, Matthew  },
year = {2021} ,
title = {IPUMS USA: Version 11.0 [dataset]. Minneapolis, MN: IPUMS} ,
institution = {University of Minnesota} ,
note = { DOI : 10.18128/D010.V11.0} 
}